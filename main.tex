\documentclass[aspectratio=149, 10pt, t]{beamer}

\renewcommand{\normalsize}{\fontsize{10.5pt}{13pt}\selectfont}

\input{preamble}
\usepackage[normalem]{ulem}

% Add separator slide to the beginning of each section ==>
%% Blue on white:
%\AtBeginSection[]{%
%	\begin{frame}[standout, c]{~}
%		%\vfill
%		\usebeamerfont{title}%
%		\textcolor{SpotColor}{\insertsectionhead}
%		%\vfill
%	\end{frame}%
%}
% White on blue:
\AtBeginSection[]{{%
	\setbeamercolor{background canvas}{bg=UBonnBlue, fg=white}%
	\colorlet{SpotColor}{white}%
	\begin{frame}[standout, c]{~}
		%\vfill
		\usebeamerfont{title}\hypersetup{linkcolor=white}%
		\insertsection
		%\vfill
	\end{frame}%
	\colorlet{SpotColor}{UBonnBlue}%
}}
%% Alternativelyy, add ``Outline'' slide to the beginning of each section ==>
%\AtBeginSection[]{
%	\begin{frame}[plain]{Outline}
%		\tableofcontents[currentsection]
%	\end{frame}
%}
%% <==

\title[La fibre, de zéro en Rézo]{%
	\textbf{\usebeamerfont{structure}%
		La fibre : de zéro en Rézo%
	}%
}

\subtitle{Un petit tour d'horizon pour n1a en détresse}

%\subtitle{I~Prefer to Avoid Subtitles}

\author{%
	Hugo Levy-Falk\\
	% Set the name of the presenting coauthor in boldface:
	Avec les conseils de Gabriel Detraz
} % Author(s)

%\institute{%
%	\inst{a}\,University of Bonn, Germany \and
%	\inst{b}\,University of Cologne, Germany \and
%	\inst{c}\,Collaborative Research Center Transregio~224%
%} % Institution(s)

\date{%
	Nocturnes FedeRez 2021 \\[\medskipamount]
	\textmd{20 Novembre 2021}%
}


\begin{document}

\begin{frame}[standout]{~}

	\titlepage%

\end{frame}


\begin{frame}[standout]{Outline}

	\medskip
	\tableofcontents

\end{frame}

\section{Introduction : C'est quoi, et pourquoi on les utilise ?}

\begin{frame}{Un canal de communication}
    \centering
    \includegraphics[height=\textheight]{images/une_fibre}
\end{frame}
\begin{frame}{Un canal de communication}
    \centering
    \includegraphics[width=0.7\linewidth]{images/Excavator_in_Brittany_France.JPG}
\end{frame}

\begin{frame}{Pour ça ?}
    \centering
    \includegraphics[width=0.7\linewidth]{images/pub_fibre.png}
\end{frame}

\begin{frame}{(Non)}
    \begin{columns}
    \begin{column}{0.5\linewidth}
        \vspace{2.25cm}
        \begin{minipage}[t]{\linewidth}
            Le signal électrique se propage à la vitesse du champ électrique dans le conducteur: \alert{$\approx 200 000\;\text{km}/\text{s}$ dans le cuivre}.
        \end{minipage}
    \end{column}
    \begin{column}{0.5\linewidth}
        \vspace{0.5cm}
        \begin{minipage}[t]{\linewidth}
            \includegraphics[width=\linewidth]{images/maxwell.jpeg}
        \end{minipage}
    \end{column}
    \end{columns}
\end{frame}

\begin{frame}{Mais pourquoi alors ?}
    \begin{columns}
    \begin{column}{0.5\linewidth}
        \medskip
        \alert{Pour un lien type ADSL}
        
        {\footnotesize(\textit{Asymmetric digital subscriber line} : demandez à vos aînés)}
        \medskip
        \begin{itemize}
            \item Sensibilité aux perturbations électromagnétiques;
            \item Atténuation pour 1km $\approx 13.81dB$ (signal divisé par 24 !)
        \end{itemize}
        \medskip
        \textbf{Conséquence :} Besoin de répétiteurs/correcteurs, débit faible : 23Mbit/s sur 1km.
    \end{column}
    \begin{column}{0.5\linewidth}
        \medskip
        \alert{Pour une fibre multimode OM2 de chez fs}
        {\footnotesize (on va voir après ce que c'est)}
        \medskip
        \begin{itemize}
            \item Pas de sensibilité aux perturbations électromagnétiques;
            \item Atténuation pour 1km $\approx 1dB$ à 1300nm (signal divisé par 1.26 (!!))
        \end{itemize}
        $\longrightarrow$ On va pouvoir diminuer les erreurs et le nombre de répétiteurs.
    \end{column}
    \end{columns}
\end{frame}

\section{Mais du coup c'est quoi une fibre ?}

\begin{frame}{Une fibre}
    \begin{columns}
        \begin{column}{0.5\linewidth}
            \begin{minipage}[b]{\linewidth}
                \centering
                \includegraphics[height=0.38\textheight]{images/fabrication_1}
                \includegraphics[height=0.38\textheight]{images/fabrication_2}

                {\footnotesize Source : J.Dutton, Ch. 6}
            \end{minipage}
        \end{column}
        \begin{column}{0.5\linewidth}
            \begin{minipage}[b]{\linewidth}
                \centering
                Loi de Snell-Descartes : $$n_1\sin\theta_1=n_2\sin\theta_2$$

                \includegraphics[height=0.3\textheight]{images/snell_descartes}

                Total Internal Reflection si : $\frac{n_1}{n_2}\sin\theta_1>1$

                \movie[loop]{\includegraphics[width=\linewidth]{videos/frame_rays.png}}{videos/rays_animation.mp4}
            \end{minipage}
        \end{column}
    \end{columns}
\end{frame}

\section{Ah, les modes de chez nous}

\begin{frame}{Une explication simple, qui marche pour les grosses fibres}
    \begin{columns}
    \begin{column}{0.5\linewidth}
            \begin{itemize}
                \item \movie[loop]{\includegraphics[width=\linewidth]{videos/frame_rays.png}}{videos/rays_animation.mp4}
                \item \movie[loop]{\includegraphics[width=\linewidth]{videos/frame_plane.png}}{videos/plane_animation.mp4}
            \end{itemize}
    \end{column}
    \begin{column}{0.5\linewidth}
        \begin{itemize}
            \item Chaque angle peut-être compris comme un "mode" de transmission.
            \item Les angles acceptables sont donnés par
            \begin{equation}
    N A=n \cdot \sin \left(\alpha_{\operatorname{Max}}\right)=\sqrt{n_{1}^{2}-n_{2}^{2}}
    \end{equation}
            \item Chaque "mode" $m$ a un indice de réfraction effectif $n_m=n_1\sin\alpha_m$. $\longrightarrow$ Première idée de dispersion (on y reviendra)
        \end{itemize}
    \end{column}
    \end{columns}
\end{frame}

\begin{frame}{La vraie vie est compliquée.}
    \begin{columns}
        \begin{column}{0.5\linewidth}
            Il faut résoudre les équations de Maxwell. On suppose $\boldsymbol{E}(x,z,t)=\boldsymbol{E}(x,t)e^{i\omega t - i\beta z}$
            \begin{itemize}
                \item Si on se place en 1D, on a
                \begin{equation}
\frac{d^{2} \boldsymbol{E}(x)}{d x^{2}}+n^{2}(x) k_{0}^{2} \boldsymbol{E}(x)=\beta^{2} \boldsymbol{E}(x)
\end{equation}
                \item Ce qui se rapproche énormément de l'équation de Schrodinger indépendante du temps !
                \begin{equation}
-\frac{\hbar^{2}}{2 m} \frac{d^{2} \psi(x)}{d x^{2}}+V(x) \psi(x)=E \psi(x)
\end{equation}
                Avec $V(x)\longrightarrow-n²(x){k_0}^2\hbar^2/2m$
            \end{itemize}
        \end{column}
        \begin{column}{0.5\linewidth}
            Et en pratique, c'est pire... Il faut résoudre en coordonnées cylindriques. 
            
            En supposant la séparation des variables ($E_z(r,\phi,z)=\mathcal{R}(r)\mathcal{F}(\phi)\mathcal{Z}(z)$), on a 3 équations qui ressortent :
            \begin{gather}
                \frac{\partial^{2} \mathcal{Z}(z)}{\partial z^{2}}=-\beta^{2} \mathcal{Z}(z)\\
                \frac{\partial^{2} \mathcal{F}(\phi)}{\partial z^{2}}=-m^{2} \mathcal{F}(\phi)\\
                \nonumber\frac{\partial^{2} \mathcal{R}(r)}{\partial r^{2}}+\frac{1}{r} \frac{\partial \mathcal{R}(r)}{\partial r}+...\\...+\left(n^{2} k_{0}^{2}-\beta^{2}-\frac{m^{2}}{r^{2}}\right) \mathcal{R}(r)=0
            \end{gather}
        \end{column}
    \end{columns}
\end{frame} 

\begin{frame}{La vraie vie est compliquée. (mais jolie)}
    Certains modes particuliers sont appelés des modes propres (en pratique on observe une combinaison de ces modes).

     Todo : Plot de quelques modes propres
    
    La fréquence de coupure (fréquence limite des modes propagés)
    
    $$V=k_0d\sqrt{{n_1}^2-{n_2}^2} = \text{NA}\frac{2\pi}{\lambda}\times d$$
    
    Nombre de modes \begin{itemize}
        \item pour du saut d'indice : $m \approx \frac{V^2}{2}$
        \item pour du gradient d'indice : $m\approx\frac{V^2}{4}$
    \end{itemize}
\end{frame}

\section{Perdez votre signal avec ces techniques secrètes (les informaticiens le détestent)}

\begin{frame}{Les facteurs de perte}
    \begin{columns}
        \begin{column}{0.5\linewidth}
        \medskip
        \alert{L'attenuation}
            
            Il y a aussi des pertes à cause :\begin{itemize}
                \item De l'absorption;
                \item Des courbures macro;
                \item Des pertes de couplages de fibres;
                \item Pertes de scattering...
            \end{itemize}
            \includegraphics[width=\linewidth]{images/different-types-of-losses-in-optical-fiber.png}
            {\footnotesize Source : \href{https://community.fs.com/fr/blog/how-to-reduce-various-types-of-losses-in-optical-fiber.html}{community.fs.com}}

        \end{column}
        \begin{column}{0.5\linewidth}
        \medskip
        \alert{La dispersion}

            Globalement, tout ce qui va faire qu'un pulse va s'élargir au cours du temps.
            TODO: refaire cette anim pour que la partie dispersée dure plus longtemps.

            \movie[loop]{\includegraphics[width=\linewidth]{videos/frame_pulses.png}}{videos/pulses_animation.mp4}
            
            $\longrightarrow$ C'est un facteur limitant du taux te transfert d'information dans une fibre. 

            \textbf{Il ne suffit pas d'augmenter la puissance d'émission pour le contrer !}
        \end{column}
    \end{columns}
\end{frame}

\begin{frame}{L'attenuation : absorption (et scattering)}
    TODO : mettre la source proprement + donner des échelles (genre on peut voir au fond de la fosse des marianes) + mettre les fenêtre télécom

    \centering
    \includegraphics[width=0.7\linewidth]{images/absorption_fibre.png}

\end{frame}

\begin{frame}{L'attenuation : les autres pertes}
    TODO : faire une table avec les facteurs de pertes (raccord, bending etc...)
\end{frame}

\begin{frame}{La dispersion}
    Remettre la vidéo
\end{frame}

\begin{frame}{La dispersion: modale}

\end{frame}

\begin{frame}{La dispersion: du matériau}
    Faire une animation qui montre que le pulse est la somme de plein de couleurs
\end{frame}
\begin{frame}{La dispersion: du guide d'onde}
\end{frame}
\begin{frame}{La dispersion: des modes de polarisation}
\end{frame}


\section{Des petits noms de fibres}

\begin{frame}{Frame Title}
-> SFP/SFP+

-> CWDM/DWDM
    
\end{frame}

\section{Quelques technologies récentes}

%\begin{frame}{Fibres photoniques}
%    
%\end{frame}

\begin{frame}{Le machin pour multiplexer les modes}
    
\end{frame}

\end{document}
